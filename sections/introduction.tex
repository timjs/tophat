% !TEX root=../icfp2019.tex



\section{Introduction}



\subsection{Tasks}

Many applications these days are developed to support workflows in institutions and businesses.
Take for example expense declarations, order processing, and emergency management.
Some of these workflows occur on the boundary between organisations and customers,
like flight bookings or tax returns.
What they all have in common is,
that they need to interact with different people (end users, tax officers, customers, etc.)
and they use information from multiple sources (input forms, databases, sensors, etc.).

We call such interactive units of work based on information sources \emph{tasks}.
Tasks model collaboration between users and are driven by work users do.
Users could be people out in the field or sitting behind their desks,
as well as machines doing calculations or fetching data.



\subsection{Task-oriented programming}

Task-oriented programming (\TOP) is a programming paradigm which searches for the sweet spot between faithful modelling workflows
and rapid prototyping of multi-user web applications supporting these workflows.
% Task-oriented programming (\TOP) is a programming paradigm to support these ways of working.
\TOP focusses on modelling collaboration patterns.
This gives rise to user's need to interact and share information.
Next to that, \TOP automatically provides solutions to common development jobs like designing \GUI\ s, connecting to databases, and communicating between servers and clients.
Therefore,
a language that supports \TOP should choose the right level of abstraction to support two things.
First, it should allow to specify tasks from real world scenarios.
Second, it should be able to generate multi-user web applications to support these scenarios.



\subsection{Utilisation}


Currently, we know of two frameworks using \TOP: \ITASKS and \MTASKS.
\ITASKS is a full fledged \TOP framework able to generate rich client and server applications from a single source in the functional programming language Clean.
\MTASKS is a subset of \ITASKS,
focussing on \IOT devices and deployment on micro controllers.
Both have been used to model real world scenarios like \fixme{add}.



\subsection{Contributions}

In this paper we tend to identify the essence of \TOP in formal as well as an informal way.
These core principles lead to a formal system for \TOP called \TOPHAT.
\TOPHAT paves the way to formal reasoning about applications written using the \TOP paradigm,
like those in \ITASKS and \MTASKS.

\TOP applications give rise to concepts like concurrency, synchronisation, \GUI-programming and program generation.
In this paper we identify common concepts with already existing languages and frameworks,
and describe their similarities and differences.



Our contributions to workflow modelling, functional programming language design, and rapid application development are as follows.

\begin{enumerate}

  \item
    We informally describe the essential concepts of task-oriented programming (\TOP).
    We have a strong focus on modelling using collaboration
    while keeping a constant desire to rapidly generate executable applications into account.

  \item
    We present a formal calculus for \TOP, arising naturally from above essential concepts.
    Hereby we provide ground work to apply formal reasoning about \TOP specifications in future work.

  \item
    Using both informal and formal descriptions, we compare \TOP and \TOPHAT with multiple related work in the area,
    ranging from business process modelling languages, to process algebras and reactive programming frameworks.

\end{enumerate}



\subsection{Structure}

\input{sections/structure}



\subsection{Old}

\input{sections/introduction-old}
