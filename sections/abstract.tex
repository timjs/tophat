% !TEX root=../icfp2019.tex

Task-Oriented Programming (\TOP) is a programming paradigm that focusses on modelling real world collaborations between people.
It prescribes a declarative programming style to specify multi-user workflows.
Workflows can be higher-order.
They communicate through typed values on a local or global level.
Such specifications can be turned into interactive applications for different platforms,
supporting collaboration during execution.

In this paper we decompose the rich features of \TOP into elementary language elements,
which makes them suitable for formal treatment.
We use the simply typed lambda calculus, extended with pairs and references, as a base language.
On top of this language, we develop \TOPHAT (TopHat), a calculus for modular interactive workflows.
With \TOPHAT we prepare a way to formally reason about \TOP systems.

We describe \TOPHAT by means of a layered semantics.
These layers consist of multiple big step evaluations on expressions,
and two labelled transition systems, handling user inputs.
We show some interesting properties of this machinery.
This approach allows for comparison with other work in the field.
We place \TOPHAT in perspective with the process calculus \CSP.

\fixme{Why?}

% Abstract
% Each submission must be accompanied by a plain-language abstract of up to 500 words that presents the key points in the paper in a manner understandable by experienced practitioners and researchers in nearby disciplines. The abstract should avoid mathematical symbols whenever possible, and it must address the following:
%
% Context: What is the broad context of the work? What is the importance of the general research area?
% Inquiry: What problem or question does the paper address? How has this problem or question been addressed by others (if at all)?
% Approach: What was done that unveiled new knowledge?
% Knowledge: What new facts were uncovered? If the research was not results oriented, what new capabilities are enabled by the work?
% Grounding: What argument, feasibility proof, artifacts, or results and evaluation support this work?
% Importance: Why does this work matter?
% The absence of an abstract conforming to this specification is grounds for the rejection of the paper without review.
%
%
